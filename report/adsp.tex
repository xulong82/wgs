%%
% Please see https://bitbucket.org/rivanvx/beamer/wiki/Home for obtaining beamer.
%%
\documentclass{beamer}
\usepackage{graphics}
\usepackage{xcolor}

\begin{document}
\setbeamertemplate{caption}{\raggedright\insertcaption\par}

\title{Novel Genomic Variants for Late-onset Alzheimer's Disease}
\author{Xulong Wang}
\date{\today}

\frame{\titlepage}
%\frame{\frametitle{Table of contents}\tableofcontents}

\section{Alzheimer's Disease} 
\subsection{Late-onset Alzheimer's Disease}

\frame{\frametitle{Late-onset Alzheimer's Disease} 
\begin{figure}
\includegraphics<1>[height=5cm]{../Figures/alzBrain.jpg}
\caption{Alzheimer's disease is the most common form of dementia affecting over 35 million people worldwide.}
\end{figure}
}

\frame{\frametitle{Late-onset Alzheimer's Disease} 
\begin{itemize}
\item Late-onset Alzheimer's disease usually occurs after age 65, and accounts for about 90\% of total AD cases. \newline
\item Top LOAD associated genes by GWAS study: \textcolor{red}{APOE $\epsilon2/3/4$, BIN1, CLU, ABCA7, CR1, PICALM, MS4A6A, CD33, MS4A4E, CD2AP} (Data from www.alzgene.org).
\end{itemize}
}
\subsection{Alzheimer's Disease Sequencing Project}

\frame{\frametitle{Alzheimer's Disease Sequencing Project}
Objective: To identify new genomic variants that increase and decrease the risk of developing AD. \newline
%\vspace{-.5cm}
\begin{figure}
\includegraphics<1>[height=2cm]{../Figures/adsp.pdf}
\end{figure}
}

\frame{\frametitle{Alzheimer's Disease Sequencing Project}
\begin{figure}
\includegraphics<1>[height=6cm]{../Figures/adPie1.pdf}
\caption{AD diagnosis for 584 people across 111 families. \color{red} Results in this presentation are analyzed from 137 people across 32 families.}
\end{figure}
}

\frame{\frametitle{Alzheimer's Disease Sequencing Project}
APOE (apolipoprotein E) on chromosome 19 makes protein that helps helps carry cholesterol in the bloodstream.
\vspace{-1cm}
\begin{columns}
\begin{column}{0.5\textwidth}
\begin{figure}
\includegraphics<1>[height=6cm]{../Figures/apoePie1.pdf}
\end{figure}
\end{column}
\hspace{-.5cm}
\begin{column}{0.6\textwidth}
\vspace{.5cm}
\footnotesize
\begin{itemize}
\item APOE $\epsilon2$ is rare and may protect against AD  
\item APOE $\epsilon3$ is the most common allele and plays a neutral role
\item APOE $\epsilon4$ present in about 25 to 30 \% of population and in about 40 \% of people with LOAD.
\end{itemize} 
\end{column}
\end{columns}
}

\frame{\frametitle{Alzheimer's Disease Sequencing Project}
\begin{columns}
\begin{column}{0.5\textwidth}
\begin{figure}
\includegraphics<1>[height=5cm]{../Figures/ageBox1.pdf}
\caption{Age distribution by AD status.}
\end{figure}
\end{column}
\begin{column}{0.5\textwidth}
\begin{figure}
\includegraphics<1>[height=5cm]{../Figures/ageHist1.pdf}
\caption{Age distribution.}
\end{figure}
\end{column}
\end{columns}
}

\frame{\frametitle{Alzheimer's Disease Sequencing Project}
\begin{figure}
\includegraphics<1>[height=6cm]{../Figures/familyBar1.pdf}
\caption{Sample number per family}
\end{figure}
}


\section{Association study} 
\frame{\frametitle{What are the genomic variants that are significant associated with AD?}
\vspace{-1cm}
\begin{eqnarray}
AD \sim Sex + Age + SNP \\
AD \sim Sex + Age + APOE + SNP \\
AD \sim Sex + Age + + SNP + (1|Family) \\
AD \sim Sex + Age + APOE + SNP + (1|Family)
\end{eqnarray}
}

\frame{\frametitle{Summary of Null Models}
\vspace{-1cm}
\begin{columns}
\begin{column}{0.5\textwidth}
\begin{figure}
\includegraphics<1>[height=6cm]{../Figures/model1.pdf}
\end{figure}
\end{column}
\begin{column}{0.5\textwidth}
\vspace{.5cm}
\footnotesize
\begin{itemize}
\item Intercepts high. Most samples are probable (0.5).  
\item APOE $\epsilon34$ drops intercepts, increases AD risk.
\item APOE $\epsilon24$ and $\epsilon33$ decrease AD risk in model4.
\item Age and Sex have little effects
\end{itemize} 
\end{column}
\end{columns}
}

\frame{\frametitle{Family level effects on intercepts}
\begin{figure}
\includegraphics<1>[height=6cm]{../Figures/model2.pdf}
\caption{Family level effects on intercepts generally small.}
\end{figure}
}

\frame{\frametitle{Summary statistics for SNP}
\vspace{-1cm}
\begin{columns}
\begin{column}{0.5\textwidth}
\begin{figure}
\includegraphics<1>[height=6cm]{../Figures/mafHist.pdf}
\end{figure}
\end{column}
\begin{column}{0.5\textwidth}
\footnotesize
\begin{tabular}{l l}
$\sim 4M$ SNPs per sample & \\
$\sim 14M$ in total & \\
& \\
\color{red} MAF $<$ .05 & \\
6690312 & 7820739 \\
& \\
\color{red} 0/1 $>$ .95 & \\
59381 & 7743948
\end{tabular}
\end{column}
\end{columns}
}

\frame{\frametitle{Permutation}
$AD \sim Sex + Age + APOE + p(SNP) + (1|Family)$
%\vspace{-1cm}
\begin{columns}
\begin{column}{0.5\textwidth}
\vspace{.5cm}
\footnotesize
\begin{enumerate}
\item Randomly pick 1M SNP.  
\item Run full model on permuted SNP.
\item Save maximal LRT value.
\item Repeat 1-3 2000 times
\item Set threshold as .95 quantile of maximal LRTs
\end{enumerate}
\end{column}
\begin{column}{0.6\textwidth}
\begin{figure}
\includegraphics<1>[height=6cm]{../Figures/lrtPmt1.pdf}
\end{figure}
\end{column}
\end{columns}
}


\frame{\frametitle{LRT line}
\begin{figure}
\includegraphics<1>[height=5.5cm]{../Figures/lineLrt1.pdf}
\caption{$AD \sim Sex + Age + APOE + SNP + (1|Family)$}
\end{figure}
}

\frame{\frametitle{LocusZoom}
\begin{figure}
\includegraphics<1>[height=5.5cm]{../Figures/rs11768450.png}
\caption{$AD \sim Sex + Age + APOE + SNP + (1|Family)$}
\end{figure}
}

\frame{\frametitle{To do list}
\begin{itemize}
\item Run models 1-4 on full data
\item Bayesian inference on multilevel ordered categorical model
\item Pair-wise regression
\end{itemize}
}

%\frame{\frametitle{table}
%\begin{tabular}{c c c}
%A & B & C \\ 1 & 2 & 3 \\  A & B & C \\ 
%\end{tabular} }
%
%\frame{\frametitle{blocs}
%\begin{block}{title of the bloc}
%bloc text
%\end{block}
%\begin{exampleblock}{title of the bloc}
%bloc text
%\end{exampleblock}
%\begin{alertblock}{title of the bloc}
%bloc text
%\end{alertblock}
%}

\end{document}
