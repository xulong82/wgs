\documentclass[11pt, landscape]{article}   	% use "amsart" instead of "article" for AMSLaTeX format
\usepackage{geometry}                		% See geometry.pdf to learn the layout options. There are lots.
\geometry{letterpaper}                   		% ... or a4paper or a5paper or ... 
%\geometry{landscape}                		% Activate for for rotated page geometry
%\usepackage[parfill]{parskip}    		% Activate to begin paragraphs with an empty line rather than an indent
\usepackage{graphicx}				% Use pdf, png, jpg, or eps§ with pdflatex; use eps in DVI mode
								% TeX will automatically convert eps --> pdf in pdflatex		
\usepackage{amssymb}

\title{Brief information of the genes}
\author{ADSP project}
%\date{}							% Activate to display a given date or no date

\begin{document}
\maketitle
%\section{}
%\subsection{}

% latex table generated in R 3.1.1 by xtable 1.7-4 package
% Thu Jan 15 14:39:59 2015
\begin{table}[ht]
%\centering
\tiny
\begin{tabular}{rlp{4cm}p{12cm}}
  \hline
 & Gene\_Name & EntrezGene\_Summary \\ 
  \hline
1 & RHD & Rh blood group, D antigen & The Rh blood group system is the second most clinically significant of the blood groups, second only to ABO. It is also the most polymorphic of the blood groups, with variations due to deletions, gene conversions, and missense mutations. The Rh blood group includes this gene, which encodes the RhD protein, and a second gene that encodes both the RhC and RhE antigens on a single polypeptide. The two genes, and a third unrelated gene, are found in a cluster on chromosome 1. The classification of Rh-positive and Rh-negative individuals is determined by the presence or absence of the highly immunogenic RhD protein on the surface of erythrocytes. Multiple transcript variants encoding different isoforms have been found for this gene. \\ 
  2 & CENPF & centromere protein F, 350/400kDa & This gene encodes a protein that associates with the centromere-kinetochore complex. The protein is a component of the nuclear matrix during the G2 phase of interphase. In late G2 the protein associates with the kinetochore and maintains this association through early anaphase. It localizes to the spindle midzone and the intracellular bridge in late anaphase and telophase, respectively, and is thought to be subsequently degraded. The localization of this protein suggests that it may play a role in chromosome segregation during mitotis. It is thought to form either a homodimer or heterodimer. Autoantibodies against this protein have been found in patients with cancer or graft versus host disease. \\ 
  3 & RYR2 & ryanodine receptor 2 (cardiac) & This gene encodes a ryanodine receptor found in cardiac muscle sarcoplasmic reticulum. The encoded protein is one of the components of a calcium channel, composed of a tetramer of the ryanodine receptor proteins and a tetramer of FK506 binding protein 1B proteins, that supplies calcium to cardiac muscle. Mutations in this gene are associated with stress-induced polymorphic ventricular tachycardia and arrhythmogenic right ventricular dysplasia. \\ 
  4 & ZNF684 & zinc finger protein 684 &  \\ 
  5 & ANKRD36 & ankyrin repeat domain 36 &  \\ 
  6 & OTOF & otoferlin & Mutations in this gene are a cause of neurosensory nonsyndromic recessive deafness, DFNB9. The short form of the encoded protein has 3 C2 domains, a single carboxy-terminal transmembrane domain found also in the C. elegans spermatogenesis factor FER-1 and human dysferlin, while the long form has 6 C2 domains. The homology suggests that this protein may be involved in vesicle membrane fusion. Several transcript variants encoding multiple isoforms have been found for this gene. \\ 
  7 & MUC4 & mucin 4, cell surface associated & The major constituents of mucus, the viscous secretion that covers epithelial surfaces such as those in the trachea, colon, and cervix, are highly glycosylated proteins called mucins. These glycoproteins play important roles in the protection of the epithelial cells and have been implicated in epithelial renewal and differentiation. This gene encodes an integral membrane glycoprotein found on the cell surface, although secreted isoforms may exist. At least two dozen transcript variants of this gene have been found, although for many of them the full-length transcript has not been determined or they are found only in tumor tissues. This gene contains a region in the coding sequence which has a variable number ($>$100) of 48 nt tandem repeats. \\ 
  8 & COMMD10 & COMM domain containing 10 &  \\ 
  9 & PAM & peptidylglycine alpha-amidating monooxygenase & This gene encodes a multifunctional protein. It has two enzymatically active domains with catalytic activities - peptidylglycine alpha-hydroxylating monooxygenase (PHM) and peptidyl-alpha-hydroxyglycine alpha-amidating lyase (PAL). These catalytic domains work sequentially to catalyze neuroendocrine peptides to active alpha-amidated products. Multiple alternatively spliced transcript variants encoding different isoforms have been described for this gene but some of their full length sequences are not yet known. \\ 
  10 & PPIP5K2 & diphosphoinositol pentakisphosphate kinase 2 & Inositol phosphates (IPs) and diphosphoinositol phosphates (PP-IPs), also known as inositol pyrophosphates, act as cell signaling molecules. HISPPD1 has both IP6 kinase (EC 2.7.4.21) and PP-IP5 (also called IP7) kinase (EC 2.7.4.24) activities that produce the high-energy pyrophosphates PP-IP5 and PP2-IP4 (also called IP8), respectively (Fridy et al., 2007 [PubMed 17690096]). \\ 
  11 & OR2J1 & olfactory receptor, family 2, subfamily J, member 1 (gene/pseudogene) &  \\ 
  12 & COG5 & component of oligomeric golgi complex 5 & The protein encoded by this gene is one of eight proteins (Cog1-8) which form a Golgi-localized complex (COG) required for normal Golgi morphology and function. The encoded protein is organized with conserved oligomeric Golgi complex components 6, 7 and 8 into a sub-complex referred to as lobe B. Alternative splicing results in multiple transcript variants. Mutations in this gene result in congenital disorder of glycosylation type 2I. \\ 
  13 & NANOS1 & nanos homolog 1 (Drosophila) &  \\ 
  14 & SPTBN2 & spectrin, beta, non-erythrocytic 2 & Spectrins are principle components of a cell's membrane-cytoskeleton and are composed of two alpha and two beta spectrin subunits. The protein encoded by this gene (SPTBN2), is called spectrin beta non-erythrocytic 2 or beta-III spectrin. It is related to, but distinct from, the beta-II spectrin gene which is also known as spectrin beta non-erythrocytic 1 (SPTBN1). SPTBN2 regulates the glutamate signaling pathway by stabilizing the glutamate transporter EAAT4 at the surface of the plasma membrane. Mutations in this gene cause a form of spinocerebellar ataxia, SCA5, that is characterized by neurodegeneration, progressive locomotor incoordination, dysarthria, and uncoordinated eye movements. \\ 
    \hline
\end{tabular}
\end{table}

\begin{table}[ht]
%\centering
\tiny
\begin{tabular}{rlp{4cm}p{12cm}}
  \hline
 & Gene\_Name & EntrezGene\_Summary \\ 
  \hline
  15 & KRT7 & keratin 7, type II & The protein encoded by this gene is a member of the keratin gene family. The type II cytokeratins consist of basic or neutral proteins which are arranged in pairs of heterotypic keratin chains coexpressed during differentiation of simple and stratified epithelial tissues. This type II cytokeratin is specifically expressed in the simple epithelia lining the cavities of the internal organs and in the gland ducts and blood vessels. The genes encoding the type II cytokeratins are clustered in a region of chromosome 12q12-q13. Alternative splicing may result in several transcript variants; however, not all variants have been fully described. \\ 
  16 & ADAMTSL3 & ADAMTS-like 3 &  \\ 
  17 & RASGRP1 & RAS guanyl releasing protein 1 (calcium and DAG-regulated) & This gene is a member of a family of genes characterized by the presence of a Ras superfamily guanine nucleotide exchange factor (GEF) domain. It functions as a diacylglycerol (DAG)-regulated nucleotide exchange factor specifically activating Ras through the exchange of bound GDP for GTP. It activates the Erk/MAP kinase cascade and regulates T-cells and B-cells development, homeostasis and differentiation. Alternatively spliced transcript variants encoding different isoforms have been identified. Altered expression of the different isoforms of this protein may be a cause of susceptibility to systemic lupus erythematosus (SLE). \\ 
  18 & NOB1 & NIN1/RPN12 binding protein 1 homolog (S. cerevisiae) & In yeast, over 200 protein and RNA cofactors are required for ribosome assembly, and these are generally conserved in eukaryotes. These factors orchestrate modification and cleavage of the initial 35S precursor rRNA transcript into the mature 18S, 5.8S, and 25S rRNAs, folding of the rRNA, and binding of ribosomal proteins and 5S RNA. Nob1 is involved in pre-rRNA processing. In a late cytoplasmic processing step, Nob1 cleaves a 20S rRNA intermediate at cleavage site D to produce the mature 18S rRNA (Lamanna and Karbstein, 2009 [PubMed 19706509]). \\ 
  19 & PKD1 & polycystic kidney disease 1 (autosomal dominant) & This gene encodes a member of the polycystin protein family. The encoded glycoprotein contains a large N-terminal extracellular region, multiple transmembrane domains and a cytoplasmic C-tail. It is an integral membrane protein that functions as a regulator of calcium permeable cation channels and intracellular calcium homoeostasis. It is also involved in cell-cell/matrix interactions and may modulate G-protein-coupled signal-transduction pathways. It plays a role in renal tubular development, and mutations in this gene cause autosomal dominant polycystic kidney disease type 1 (ADPKD1). ADPKD1 is characterized by the growth of fluid-filled cysts that replace normal renal tissue and result in end-stage renal failure. Splice variants encoding different isoforms have been noted for this gene. Also, six pseudogenes, closely linked in a known duplicated region on chromosome 16p, have been described. \\ 
  20 & CIRH1A & cirrhosis, autosomal recessive 1A (cirhin) & This gene encodes a WD40-repeat-containing protein that is localized to the nucleolus. Mutation of this gene causes North American Indian childhood cirrhosis, a severe intrahepatic cholestasis that results in transient neonatal jaundice, and progresses to periportal fibrosis and cirrhosis in childhood and adolescence. \\ 
  21 & CCL4L2 & chemokine (C-C motif) ligand 4-like 2 &  \\ 
  22 & LRRC8E & leucine rich repeat containing 8 family, member E & This gene encodes a member of a small, conserved family of proteins with similar structure, including a string of extracellular leucine-rich repeats. A related protein was shown to be involved in B-cell development. Alternatively spliced transcript variants encoding multiple isoforms have been observed for this gene. \\ 
  23 & AC010336.1 &  &  \\ 
  24 & GCNT7 & glucosaminyl (N-acetyl) transferase family member 7 &  \\ 
  25 & MYH7B & myosin, heavy chain 7B, cardiac muscle, beta & The myosin II molecule is a multi-subunit complex consisting of two heavy chains and four light chains. This gene encodes a heavy chain of myosin II, which is a member of the motor-domain superfamily. The heavy chain includes a globular motor domain, which catalyzes ATP hydrolysis and interacts with actin, and a tail domain in which heptad repeat sequences promote dimerization by interacting to form a rod-like alpha-helical coiled coil. This heavy chain subunit is a slow-twitch myosin. Alternatively spliced transcript variants have been found, but the full-length nature of these variants is not determined. \\ 
    \hline
\end{tabular}
\end{table}

\end{document}  

